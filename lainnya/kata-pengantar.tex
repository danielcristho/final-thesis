\begin{center}
  \Large
  \textbf{KATA PENGANTAR}
\end{center}

\addcontentsline{toc}{chapter}{KATA PENGANTAR}

\vspace{2ex}

% Ubah paragraf-paragraf berikut dengan isi dari kata pengantar

Puji dan syukur kehadirat Tuhan Yang Maha Esa yang memberikan karunia, rahmat, dan pertologan sehingga penulis dapat menyelesaikan penelitian tugas akhir yang berjudul 'PENGELOLAAN PENGGUNAAN INFRASTRUKTUR GPU UNTUK PENGGUNA BERBASIS DOCKER CONTAINER MENGGUNAKAN JUPYTERLAB'. Melalui kata pengantar ini, penulis mengucapkan terima kasih sebesar-besarnya kepada seluruh pihak yang telah membantu dan mendukung penulis selama mengerjakan penelitian tugas akhir ini, diantarnya adalah:

\begin{enumerate}[nolistsep]

  \item Tuhan Yang Maha Esa, atas karunia dan rahmat-Nya sehingga penulis dapat mencapai titik akahir perkuliahan strata satu di Departemen Teknik Informatika, Institut Teknologi Sepuluh Nopember.

   \item Kedua orang tua yang telah mendukung penulis selama berkuliah di Departemen Teknik Informatika, Institut Teknologi Sepuluh Nopember.

  \item Bapak Ir. Ary Mazharuddin Shiddiqi, S.Kom., M.Comp.Sc., Ph.D. dan Bapak Royyana Muslim Ijtihadie, S.Kom., M.Kom., Ph.D. sebagai dosen pembimbing yang telah membimbing, memberi arahan, dan masukan kepada penulis selama mengerjakan tugas akhir ini.

  \item Dosen dan tenaga pendidik di Departemen Teknik Informatika, Institut Teknologi Sepuluh Nopember yang telah memberikan pengetahuan, wawasan, dan pengalaman yang sangat berarti selama masa studi.

  \item Pihak-pihak lain yang tidak dapat disebutkan satu persatu yang telah membantu penulis dalam pelaksanaan penelitian tugas akhir ini.

\end{enumerate}

Akhir kata, semoga penelitian tugas akhir ini dapat memberikan kontribusi yang bermanfaat. Terima kasih dan permohonan maaf atas kekurangan dan kesalahan dalam pelaksanaan tugas akhir ini.

\begin{flushright}
  \begin{tabular}[b]{c}
    \place{}, \MONTH{} \the\year{} \\
    \\
    \\
    \\
    \\
    \name{}
  \end{tabular}
\end{flushright}
