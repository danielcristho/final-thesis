% Atur variabel berikut sesuai namanya

% nama
\newcommand{\name}{Gloriyano Cristho Daniel Pepuho}
\newcommand{\authorname}{Pepuho, Gloriyano Cristho Daiel}
\newcommand{\nickname}{Daniel}
\newcommand{\advisor}{Ir. Ary Mazharuddin Shiddiqi, S.Kom., M.Comp.Sc., Ph.D}
\newcommand{\coadvisor}{Royyana Muslim Ijtihadie, S.Kom., M.Kom., Ph.D.}
\newcommand{\examinerone}{Dr. Galileo Galilei, S.T., M.Sc}
\newcommand{\examinertwo}{Friedrich Nietzsche, S.T., M.Sc}
\newcommand{\examinerthree}{Alan Turing, ST., MT}
\newcommand{\headofdepartment}{Prof. Albus Percival Wulfric Brian Dumbledore, S.T., M.T}

% identitas
\newcommand{\nrp}{5025201121}
\newcommand{\advisornip}{19810620 200501 1 003}
\newcommand{\coadvisornip}{19770824 200604 1 001}
\newcommand{\examineronenip}{18560710 194301 1 001}
\newcommand{\examinertwonip}{18560710 194301 1 001}
\newcommand{\examinerthreenip}{18560710 194301 1 001}
\newcommand{\headofdepartmentnip}{18810313 196901 1 001}

% judul
\newcommand{\tatitle}{PENGELOLAAN PENGGUNAAN INFRASTRUKTUR GPU UNTUK PENGGUNA BERBASIS DOCKER CONTAINER MENGGUNAKAN JUPYTERLAB}
\newcommand{\engtatitle}{\emph{Managing Distributed GPU Infrastructure Usage for Users Based on
Docker Containers Using JupyterLab}}

% tempat
\newcommand{\place}{Surabaya}

% jurusan
\newcommand{\studyprogram}{Departemen Teknik Informatika}
\newcommand{\engstudyprogram}{Department of Informatics Engineering}

% fakultas
\newcommand{\faculty}{Fakultas Fakultas Teknologi Elektro dan Informatika Cerdas}
\newcommand{\engfaculty}{Faculty of Intelligent Electrical and Informatics Technology}

% singkatan fakultas
\newcommand{\facultyshort}{FTEIC}
\newcommand{\engfacultyshort}{FTEIC}

% departemen
\newcommand{\department}{Departemen Teknik Informatika}
\newcommand{\engdepartment}{Department of Informatics Engineering}

% kode mata kuliah
\newcommand{\coursecode}{EF234801}
