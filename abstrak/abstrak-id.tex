\begin{center}
  \large\textbf{ABSTRAK}
\end{center}

\addcontentsline{toc}{chapter}{ABSTRAK}

\vspace{2ex}

\begingroup
% Menghilangkan padding
\setlength{\tabcolsep}{0pt}

\noindent
\begin{tabularx}{\textwidth}{l >{\centering}m{2em} X}
  Nama Mahasiswa    & : & \name{}         \\

  Judul Tugas Akhir & : & \tatitle{}      \\

  Pembimbing        & : & 1. \advisor{}   \\
                    &   & 2. \coadvisor{} \\
\end{tabularx}
\endgroup

% Ubah paragraf berikut dengan abstrak dari tugas akhir
Dalam era teknologi yang semakin maju, kebutuhan akan komputasi berbasis GPU menjadi
sangat penting, khususnya dalam bidang kecerdasan buatan (AI) dan analisis data skala besar. GPU memungkinkan pemrosesan paralel yang cepat dan efisien, sehingga sering digunakan untuk melatih model deep learning dan menjalankan tugas-tugas komputasi intensif. Namun, pengelolaan GPU di lingkungan multi-pengguna menghadapi tantangan besar, seperti alokasi sumber daya yang tidak merata dan potensi penurunan efisiensi sistem. Untuk mengatasi masalah ini, penelitian ini bertujuan untuk mengembangkan mekanisme penjadwalan GPU yang efisien dengan memanfaatkan teknologi Docker Container dan antarmuka JupyterLab. Docker digunakan untuk menciptakan lingkungan kerja yang terisolasi bagi setiap pengguna, sementara
JupyterLab menyediakan platform interaktif yang memudahkan pengguna dalam mengakses dan menjalankan tugas berbasis GPU secara simultan. Penelitian ini dibagi kedalam beberapa tahap yang meliputi analisis kebutuhan, desain sistem, serta perancangan metode evaluasi. Rancangan sistem yang diusulkan akan diimplementasikan pada klaster GPU di lingkungan laboratorium atau institusi pendidikan. Evaluasi direncanakan mencakup pengujian efisiensi alokasi sumber daya, kemudahan akses pengguna, dan skalabilitas sistem dalam mendukung banyak pengguna secara bersamaan. Penelitian ini diharapkan dapat memberikan kontribusi terhadap pengelolaan sumber daya GPU dalam lingkungan komputasi terdistribusi, mendukung efisiensi dan keadilan alokasi, serta meningkatkan pengalaman pengguna dalam mengakses sumber daya GPU untuk kebutuhan komputasi modern.

% Ubah kata-kata berikut dengan kata kunci dari tugas akhir
\textbf{Kata Kunci: \textit{Klaster GPU, Docker Container, JupyterLab, Pengelolaan pengguna}}