\begin{center}
  \large\textbf{ABSTRACT}
\end{center}

\addcontentsline{toc}{chapter}{ABSTRACT}

\vspace{2ex}

\begingroup
% Menghilangkan padding
\setlength{\tabcolsep}{0pt}

\noindent
\begin{tabularx}{\textwidth}{l >{\centering}m{3em} X}
  \emph{Name}     & : & \name{}         \\

  \emph{Title}    & : & \engtatitle{}   \\

  \emph{Advisors} & : & 1. \advisor{}   \\
                  &   & 2. \coadvisor{} \\
\end{tabularx}
\endgroup

% Ubah paragraf berikut dengan abstrak dari tugas akhir dalam Bahasa Inggris
In the era of advancing technology, the demand for GPU-based computing has become
increasingly critical, particularly in the fields of artificial intelligence (AI) and large-scale data analysis. GPUs enable fast and efficient parallel processing, making them widely used for training deep learning models and performing computationally intensive tasks. However, managing GPUs in multi-user environments presents significant challenges, such as uneven resource allocation and potential system inefficiencies. To address these issues, this study aims to develop an
efficient GPU scheduling mechanism utilizing Docker container technology and the JupyterLab interface. Docker creates isolated work environments for each user, while JupyterLab provides an interactive platform that simplifies simultaneous GPU-based task execution. The research consists of several phases, including requirement analysis, system design, and evaluation method planning. The proposed system design will be implemented on a GPU cluster in a laboratory or educational institution environment. Evaluation will include testing resource allocation
efficiency, user accessibility, and system scalability in supporting multiple concurrent users. This study is expected to make a significant contribution to GPU resource management in distributed computing environments, promoting efficiency and fairness in resource allocation while enhancing the user experience in accessing GPU resources for modern computational needs.

% Ubah kata-kata berikut dengan kata kunci dari tugas akhir dalam Bahasa Inggris
\textbf{\textit{Keywords: GPU Cluster, Docker Container, JupyterLab, User Management}}
